%!TEX root = ../main.tex
The maximum flow problem is an optimization problem which take place in the graph theory. This problem is noteworthy by the long succession of research contributions that have improved on the worst-case complexity of the best known algorithms. Among those best known algorithms, we can cite, in the order of creation, the Ford-Fulkerson algorithm (1955), the blocking flow of Dinitz (1970), the Edmonds-Karp algorithm (1972), the push-relabel algorithm of Goldberg and Tarjan (1986) and the binary blocking flow of Goldberg and Rao (1997). \\

The goal of this master thesis is to do an analysis of how the augmenting path algorithms and the preflow-push algorithm, two families of maximum flow algorithm, perfoms in different families of graphs. We also studied which data structure was the most suited to represent a graph. An other goal of this work is to present an open-source and modulable Java implementation of these algorithms. With this implementation, we want to allow other developpers to try our implementation and extend it. \\

Our work is divided into several parts. First, we introduce the problem studied and we define the notations that we will use throughout this work. Then we will define the algorithms and data structures that we used for our analysis. After, we show the improvements on these algorithms to make them more efficient. Then will come the part where we explain our choices and the structure of the implementation. Finally, the experimental analysis will conclude our work.\\



