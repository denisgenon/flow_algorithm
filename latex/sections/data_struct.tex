%!TEX root = ../main.tex
\section{Introduction}
For this thesis, we wanted to be able to analyse the difference between several data structures. Moreover, willing to work with big graph, the traditional adjacency matrix became too heavy. So we decide to use a structure defined below. \newline

Like for the adjacency matrix, we use a array where each row represents the neighbours of a node. For example, the first row contains the information on the neighbours of the node 0. But contrary to the adjacency matrix, we do not use a array to represent neighbours but a different structure requiring less memory space. \newline

We used four various data structures : Hash Map, Tree Map, Simple Linked List and one home-made structure, Split Array. Each node will have its structure, storing which nodes are neighbours and what are the capacities of the edges of these nodes. If we had to represent a graph with 10 nodes, we would have a array of 10, for example, Hash Map. Every Hash Map representing the neighbourhood of a single node.

\section{Data structures}
\subsection{Hash Map}
A hash map is an unordered associative array, associates a key with a value, so use as little space as possible. It contains an single array of buckets, where the values are stored. A hash function converts the key into index, which represents the bucket where the record (key/value) is stored. \newline

Ideally, the hash function assigns to every key a different bucket but it is possible to have several keys giving the same hash code. This is called a \textit{collision}. The bucket can thus contain several records. \newline

The \textit{load factor} is the number of records divided by the number of buckets.  The more the load factor is high, the more the hash map is slow. But having a too low load factor does not save search time, it just uses some memory pointlessly. To keep the load factor to a defined value (eg between $2/3$ and $3/4$), we must, when inserting new records, resize the hash map. \newline

TODO Je met un exemple de hash map? \newline

In our case, the key is the id of the nearby node and the value is the capacity of the edge.
\subsection{Tree Map}

\subsection{Simple Linked List}

\subsection{Split Array}

\subsection{Complexities}
