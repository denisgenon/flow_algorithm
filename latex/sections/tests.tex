%!TEX root = ../main.tex
This chapter presents the tests made on the framework and their results. First, the TSP is tested, which is the base case of the framework. Next, a classical CVRP implementation and a VRPTW implementation are tested. After that, an example of CVRP with new constraints relative to the client is tested, and finally, an example with path constraint is tested.
These tests have two goals, the first is to compare the efficiency of the ACO implementation, the second is to test the framework by giving tests examples on extended CVRP (CVRP with new constraints). To achieve that, the tests are run with a configuration that is a good compromise between the computation time and the solution quality. The configuration is the value  that is given to different constants like the number of ants, iterations ,$\alpha$, $\beta$ ... The values given have been found in the literature and by testing.

All the tests are made on a personal computer with Intel Core i7-3610 CPU that runs at 2.30 GHz.

\section{The Traveling Salesman Problem}
These first series of tests permit to determine the efficiency of the base implementation of the framework. The tests are made with problems of different sizes that we can find in TSPLIB \cite{tsplib}. Four different tests of different size are made. 
The TSP problems that are used are :
\begin{description}
\item[eil51] A TSP with 51 cities by Christofides
\item[berlin52] A TSP that represents 52 locations in Berlin.
\item[eil76] A TSP with 76 cities by Christofides 
\end{description}

These instances of TSP are chosen over the others from the TSPLIB because they have been tested with other ACO implementations and thus we can compare similar results together. Each instance of the TSP problem is tested a hundred times.
In table \ref{tab:mytsp}, the results of the tests are listed. In table \ref{tab:acstsp}, the results obtained by two other implementations \cite{dorigo1997acs} \cite{hlaing2011ant} are presented. In the implementation \cite{dorigo1997acs} 1250 iterations and a number $N$ of ants that is equal to the number of cities in the problem are used. No similar information has been found for the implementation \cite{hlaing2011ant}.

As a reminder, here is the signification of the names of the columns:
\begin{description}
\item[\#iterations] denotes the number of iterations made.
\item[\#ants] denotes the number of ants used.
\item[$\alpha$] is a parameter influencing the use of the pheromone
\item[$\beta$] is a parameter influencing the importance of the heuristic function.
\item[mean] is the mean value of the path length of the 100 trials.
\item[best] is the best value of the path length of the 100 trials.
\end{description}

\begin{table}%
\centering
\small
\begin{tabular}{|l|l|l|l|l|l|l|}
\hline
problem & \#iterations & \#ants & $\alpha$ & $\beta$ & mean & best \\
\hline
\hline
eil51 & 500 & 51  & 5.0 & 5.0 & 451.48 & 430.34  \\
\hline
berlin52 & 500 & 52  & 5.0 & 5.0& 7730.00 & 7544.36\\
\hline
eil76 & 500 & 76 & 5.0 & 5.0 & 592.27 & 567.96  \\
\hline
\end{tabular}
\caption{results of the tests for the four TSP instances}
\label{tab:mytsp}
\end{table}



\begin{table}%
\centering
\begin{tabular}{|l|l|l|}
\hline
Implementation & problem & best \\
\hline
ACS \cite{dorigo1997acs} & eil51 & 427.96\\
& eil76 & 542.37 \\
ACO for TSP \cite{hlaing2011ant} & eil51 & -- (426) \\
& berlin52 & 7544.36 (7542)\\
& eil76 & -- (538) \\ 
\hline
\end{tabular}
\caption{results of other ACO implementation for the TSP}
\label{tab:acstsp}
\end{table}

The values that are in brackets are rounded values. They are calculated with rounded distances.

The results of the TSP implementation are quite good. Compared with the ACS algorithm developed by M.Dorigo and L.M.Gambardella, the results are almost the same. The same goes for the implementation of Hlaing and Khine.

Tests are not made on TSP instances with more cities because, in order to have good results, new functions have to be made to improve the results. For example, a local search method or a structure known as candidate list must be implemented. A candidate list is a preselection of nodes that can be elected at each step by an ant. With the framework, if these functions are encoded, it can be easily added by modifying the class \texttt{MyAnt}.

\section{The Capacitated Vehicle Routing Problem}\label{testcvrp}
As for the TSP, the instances of CVRP that are tested are choosen so that they can be compared to other ACO implementations. The instances that are used come from the problem described by Christofides, Mingozzi and Toth. In particular their problem: 
\begin{description}
	\item [C1] A CVRP with 51 cities and a truck capacity of 160.
	\item [C3] A CVRP with 101 cities and a truck capacity of 200.
	\item [C4] A CVRP with 151 cities and a truck capacity of 200.
\end{description}
The results of the framework are compared with the results of J.Bell and P.McMullen \cite{bell2004ant} and Y.Bin, Y.Zhong-Zhen and Y.Baozhen \cite{yu2009improved}. These two tests are run with a number of 5000 iterations and a fixed number of 30 ants. In table \ref{tab:mycvrp} are listed the results obtained with the ACO framework and in table \ref{tab:acocvrp} the results of the scientists given above.

\begin{table}[h]
\centering
\small
\begin{tabular}{|l|l|l|l|l|l|l|}
\hline
problem & \#iteration & \#ants & $\alpha$ & $\beta$ & mean & best \\
\hline
\hline
C1 & 500 & 51 & 5.0 & 5.0 & 587.23 & 556.75  \\
\hline
C3 & 500 & 100 & 5.0 & 5.0 & 983.26 & 942.38  \\
\hline
C4 & 500 & 100 & 5.0 & 5.0 & 1272.34 & 1269.65 \\
\hline

\end{tabular}
\caption{results of the tests for the CVRP}
\label{tab:mycvrp}
\end{table}

\begin{table}[h]
\centering
\begin{tabular}{|l|l|l|l|}
\hline
implementation & problem & mean & best \\
\hline
\hline
\cite{bell2004ant} & C1 & 528.90 & 524.80 \\
& C3 & 865.65 & 854.20 \\
& C4 & 1143.43 & 1131.83\\
\hline
\cite{yu2009improved} & C1 & 524.61 & 524.61\\
& C3 & 844.32 & 830.00 \\
& C4 & 1042.52 & 1028.42\\
\hline
%\cite{tan2012ant} & C1 &  \\ %% pourcentages a transformer. a faire..
%\hline
\end{tabular}
\caption{results of other ACO implementation of CVRP}
\label{tab:acocvrp}
\end{table}
The solutions provided by the framework are within $10\%$ of the optimum of other ACO algorithms. This can be explained by several factors. First, the number of iterations used is much greater in the implementations \cite{bell2004ant} \cite{yu2009improved} where they set the number of iterations to $5000$ than here in the tests where we set the number of iterations to $500$. Next, these implementations have special local search algorithms and use candidate lists to improve the results, the framework implementation tested here does not (but again, if the functions are encoded, the framework can easily be extended). Finally, the test with 151 cities is already quite large for a basic implementation thus it is normal to observe bad results.


\section{Vehicle Routing Problem with Time Windows}
The VRPTW benchmark tested are instances of the Solomon benchmark \cite{solomon}. The implementation made in the framework is a simplified version of the MACS-VRPTW \cite{gambardella1999macs}. The results obtained are compared to the original MACS-VRPTW version. In the simplified version, we omit the local search function and an insertion function that are implemented in the MACS-VRPTW. The results obtained with the framework are presented in table \ref{tab:myvrptw}, the ones obtained with the MACS implementation are presented in table \ref{tab:macsvrptw}. Values in brackets represent the number of vehicles used. The number of iterations used by \cite{gambardella1999macs}  is fixed with a computational time limit. As it is implemented in C++ and uses multi-threading at its maximum, the comparaison is hard to make.

\begin{table}[h!]
\centering
\small
\begin{tabular}{|l|l|l|l|l|l|l|l|}
\hline
problem & \#iterations & \#ants & $\alpha$ & $\beta$ & mean  & best \\
\hline
\hline
R101.25 & 500 & 25 & 5.0 & 5.0 & 889.51 (10) & 802.95 (10)  \\
\hline
R101.50 & 500 & 50 & 5.0 & 5.0 & 1567.08 (18) & 1446.07 (18)  \\
\hline
\end{tabular}
\caption{results of the tests for the VRPTW}
\label{tab:myvrptw}
\end{table}

\begin{table}[h!]
\centering
\begin{tabular}{|l|l|l|}
\hline
implementation & problem & best  \\
\hline
\hline
MACS-VRPTW & R101.25 & 617.1 (8) \\
& R101.50 & 1044.0 (12) \\
\hline
\end{tabular}
\caption{results of the MACS implementation of VRPTW}
\label{tab:macsvrptw}
\end{table}

As can be seen, the simplified implementation of the framework has bad results. This can be explained by the fact that the insertion function plays an important role in the construction of solutions that minimize the number of vehicles used. That function was not clearly described in the paper of Gambardella, but again, if we have that function, it can be easily added to the framework and thus improve the results.

\section{CVRP with in-path constraints}
This section will present two CVRP problems with new constraints added in the construction of a solution. These constraints are called "in-path" constraints because they have an effect on the construction of the solution by the ant. This type of constraint is opposed to the "off-path" type where a path is validated or not based on the finished path. 

The first test simulates a constraint on the truck. The trucks used for the vehicle problem have a limited fuel capacity. Thus a sub-path can not have a length greater than a certain value $tour\_length$. The adding of that constraint is done as explained in section \ref{howtoframework}. 

The second test simulates a preference of the clients who want to be first in a tour. Each client is assigned a value $pref$ from $1$ to $10$. Clients with high value are clients that need to be first in a tour and clients with low values are clients that do not want to be first in the tour. Both tests are performed on the C1 instance of the CVRP presented in section \ref{testcvrp}. The results of these tests are listed in the table \ref{tab:vrpin}. 


\paragraph{Limited subpath} For the test, the maximum subpath is set to 50. In the best case there are eleven vehicles. The best result of the problem C1 given in section \ref{testcvrp} had 10 vehicles and a total length of 556. Here with a limitation of the subtours of 50, with a simple calculation, it points out that at least 11 vehicles are needed. The length of the complete path is almost doubled. This can be explained by the fact that when new vehicle are used, there are more trips between the depot and the cities. An analysis of the different paths given by the 100 executions of the problem reveals that all the constraints are always verified as expected.

\paragraph{Client preference} The constraint has been implemented in a soft way. The goal here is to satisfy the most clients but it is not always possible. The objective function that gives a score to a solution has thus been modified (see section \ref{inconstraints}). The score takes into account the total distance (as in the usual CVRP) and the respect of the preference. The best path and the preference associated with client of interest are given in table \ref{tab:path} (as a reminder 0 represent the depot).  As can be seen in that table, the first clients of each tour have high preferences. The other nodes and preferences of each subtour are not reported as the preference is only taken into account for the first client visited after a depot.
\begin{table}
\centering
\small
\begin{tabular}{|l|l|l|l|l|l|l|l|}
\hline
problem & \#iterations & \#ants & $\alpha$ & $\beta$ & mean & best  \\
\hline
\hline
limited sub-path & 500 & 50 & 5.0 & 5.0 & 919.66 & 881.37 (18) \\
\hline
First client pref. & 500 & 50 & 5.0 & 5.0 & 710.14 & 614.00  \\
\hline
\end{tabular}
\caption{results of the CVRP implementation with in-path constraints}
\label{tab:vrpin}
\end{table}

\begin{table}
\centering
\begin{tabular}{|l|l|}
\hline
best path & preference \\
\hline
\hline
 0 & \\
 46 & 6\\
 47 & \\
 ... &\\
 0 & \\
 48& 7\\
 8& \\
 ...&\\
 0 & \\
 17 &10 \\
 37& \\
 ...&\\
 0 & \\
 4 & 10 \\
 44& \\
 ...&\\
\hline
\end{tabular}
\caption{results of the CVRP implementation with in-path constraints}
\label{tab:path}
\end{table}

This test shows that it is possible to add a constraint on the path in an easy way. It also shows that the constraint tends to be respected. The path is different from the classic CVRP presented earlier, and the constraint added modifies the path. In table \ref{tab:path}, we can see that the first cities of a tour have a high preference as expected.

\section{CVRP with off-parcour constraints}
The "off-parcour" constraints are constraints that can only be verified after the construction of a tour. As for example in a VRP with 3D-loading constraints where some rules must be verified to validate a path. Here a simple rule is implemented but this simple rule can be replaced by a more complex one. The rule forbids two cities having consecutive numbers to be visited one after the other. In other words, a city $i$ can not be visited after city $i-1$ or city $i+1$. A simple function verifies a path and indicates if it is valid or if it must be discarded. The result of that implementation on the C1 CVRP instance is given in table \ref{tab:cvrpoff}. An analysis on the path given after the execution of the test proves that the constraint is verified. The best result is different of the normal C1 instance because, in the best solution of that problem, there are several consecutive nodes.

Interestingly, the problem has been modified with a new constraint and so the result slightly differs from the original instance of the CVRP. It shows again that the constraint added have an effect on the results produced and are respected.

\begin{table}
\centering
\small
\begin{tabular}{|l|l|l|l|l|l|l|l|}
\hline
problem & \#iterations & \#ants & $Q_0$ & $\alpha$ & $\beta$ & mean & best  \\
\hline
\hline
off-parcour & 500 & 50 & 0.9 & 5.0 & 5.0 & 634.09 & 601.55  \\
\hline
\end{tabular}
\caption{results of the CVRP implementation with off-path constraints}
\label{tab:cvrpoff}
\end{table}