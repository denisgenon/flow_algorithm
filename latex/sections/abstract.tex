%!TEX root = ../main.tex
Ant Colony Optimization is a metaheuristic which mimics ants behavior to solve optimization problems. Ant Colony Optimization has proven to be very effective in various fields of applications. One inconvenient with ACO is the complex mapping of a problem into something that can be used by this metaheuristic. Also, adapting an ACO application from one problem to another is difficult and requires a lot of programming efforts. 

The goal of this thesis is to build an application easing the use of ACO in routing problems by creating an implementation with a high level of abstraction. This is done through the building of a framework in Scala. The result is a framework that you can use to solve different vehicle routing problems. The framework permits to add specific constraints in vehicle routing problems in a quite easy way, as presented in this thesis.

In this research, tests were made on traveling salesman problems, capacitated vehicle routing problems, vehicle routing problems with time windows and capacitated vehicle routing problems with specific constraints. The results obtained with the framework are encouraging as in almost all cases they are comparable to other ACO implementations solving these problems.