%!TEX root = ../main.tex

After all our work, we can draw several conclusions. First, which data structure should we use for each algorithm? Clearly, in view of the results, the data structures based on the sparse set are most suitable for maximum flow algorithms. Indeed, we have, in our analysis, focused attention on the importance of the functions $getAdjacents$ (which return the set of neighbours contained in the structure) and $getCapacity$ (which return the edge's capacity to a neighbour). These are the two major functions used in these algorithms. Ford-Fulkerson with scaling prefer a data structure with fast function $getCapacity$ since it makes tremendous appeal to it due to its scaling. Edmonds-Karp and the pre-flow algorithms behave in a similar way from the point of view of the data structure, they are more efficient with structure adapted to the function $getAdjacents$.
\\

Depending on a graph, which algorithm should we use? One of the first conclusion is that in most cases, if you have to choose among the augmenting path algorithms, select Edmonds-Karp. Its performances are better than Ford-Fulkerson with scaling. We believe that the heuristic used in Edmonds-Karp, which is to send first the flow on the shortest augmenting path, is suitable and effective for solving the maximum flow problem. What about pre-flow algorithms? Our master thesis highlighted two things, the pre-flow algorithms can be extremely fast but are not regular at all. Indeed, if the graph allows the ping pong effect, Edmonds-Karp will perform much better than Push-Relabel. So if your graph has a random structure, we recommend to use Edmonds-Karp while if you know that the ping pong effect is not possible on your graph, Push-Relabel offer performances that Edmonds-Karp can not compete.