%!TEX root = ../main.tex
\section{Ford-Fulkerson Scaling}
The Ford-Fulkerson algorithm has a pseudo-polynomial time complexity but there is a variant of this algorithm which has a polynomial time complexity. It is the Ford-Fulkerson with scaling algorithm. The scaling consists in looking for an augmenting path which has a large enough residual capacity. To do it, we use $G_\Delta$ which is $G$ with only the edges having a capacity $\geq \Delta$. \\

As long as there is an augmenting path in $G_\Delta$, we send flow through it. This is a $\Delta$-scaling phase. When a $\Delta$-scaling phase is ended, we divide $\Delta$ by 2 and begin the next $\Delta$-scaling phase. Initially, $\Delta = U$, with $U$ the maximum capacity of the graph.

\subsection{Complexity}
To compute the complexity of the Ford-Fulkerson Scaling algorithm, we need to know how many $\Delta$-scaling phases are possible, how many augmenting paths can be discovered at most in a $\Delta$-scaling phase and how an augmenting path is found.

\begin{itemize}
\item There is at most $O(log(U))$ $\Delta$-scaling phases because initially $\Delta = U = 2^{log(U)}$ and after each phase, $\Delta=\frac{\Delta}{2}$.
\item There is at most $O(|E|)$ augmenting paths in each $\Delta$-scaling phase. To prove it, let $f$ be the flow at the end of the $\Delta$-scaling phase and $f^*$ be the maximal flow. At the end of a $\Delta$-scaling phase, the total flow which we can add to $f$ to obtain $f^*$ is $\leq |E|*\Delta$. Since we know that in the next $\Delta$-scaling phase, $\Delta=\frac{\Delta}{2}$, there will be a maximum of $2|E|$ augmentations during the next phase.
\item We know that each augmenting path can be found in $O(|E|)$
\end{itemize}

The Ford-Fulkerson Scaling algorithm is thus bounded by $O(|E|^2* log(U))$.

\section{Push relabel}
